\documentclass[a4paper,12pt,twoside]{article}

\usepackage[brazil]{babel}
\usepackage{poliusp} 
\usepackage{amsmath,amssymb}
% Apenas preencha os dados aqui:
\titulo{Relatório de Sistemas Digitais}
\subtitulo{Laboratório 4: Simulação do Processador LEGv8}
\Autores{Nome do Aluno - NUSP XXXXXXX}
\departamento{Departamento de Computação e Sistemas Digitais}
\data{\today}


\begin{document}
% A CAPA APARECERÁ AUTOMATICAMENTE AQUI
\tableofcontents 
\clearpage

\section{Seção com Matemática}

Texto com matemática inline: $e^{i\pi}+1=0$, $\alpha+\beta=\gamma$ e $x^2+y^2=z^2$.

\subsection{Ambiente equation}

\begin{equation}
  \int_{0}^{\infty} e^{-x^2}\,dx = \frac{\sqrt{\pi}}{2}.
\end{equation}

\subsection{Ambiente align}

\begin{align}
  (a+b)^2 &= a^2 + 2ab + b^2, \\
  (a-b)^2 &= a^2 - 2ab + b^2.
\end{align}

\section{Seção com Casos e Matrizes}

\subsection{Função definida por partes}

\begin{equation}
  f(x) =
  \begin{cases}
    x^2, & x \ge 0,\\
    -x,  & x < 0.
  \end{cases}
\end{equation}

% Uso: \tabelaSimulacao{Legenda}{Label}{Conteúdo das Linhas}

\newpage

\subsection{Matriz e vetor}

\[
A =
\begin{pmatrix}
1 & 2 & 3\\
0 & 1 & 4\\
0 & 0 & 1
\end{pmatrix},
\qquad
b =
\begin{pmatrix}
1\\
0\\
-1
\end{pmatrix}.
\]

\section{Seção com Símbolos,Notação e Tabela}

%gera uma caixa pequena de sumario das subsection da section atual
\gerarSumarioLocal

Texto com símbolos: $\forall x \in \mathbb{R}$, $\exists y>0$ tal que $|x|<y$ e
$\sum_{k=0}^{n} k = \frac{n(n+1)}{2}$.

\[
\lim_{n\to\infty}\left(1+\frac{1}{n}\right)^n = e.
\]

\subsection{Tabela de Simulação labdig}

%exemplos de atalho para tabelas, foi pensando para uso pessoal, por isso a escolhas b
% Parâmetros: #1=Legenda, #2=Label, #3=Conteúdo das linhas
\tabelaSimulacao
{Cenário 1 -- Acerto das 16 jogadas (Simulação)}
{tab:cenario1}
{
    1  & Resetar circuito & \texttt{reset\_in = 1} & FSM no estado inicial & \\
    2  & Aguardar estabilidade & - & - & \\
    3  & Acionar iniciar & \texttt{iniciar\_in = 1} & Aguardando jogada 1 & Aguardando jogada 1 \\
    % ... adicione as outras linhas aqui ...
    19 & Acionar jogada 16 & \texttt{chaves = 4'b0100} & 
    \begin{tabular}[c]{@{}l@{}}Sinais \textbf{acertou} e \\ \textbf{pronto} ativados\end{tabular} & \\
}
\subsection{tabela de Duas Colunas}

% \tabelaDuasColunas{<caption>}{<label>}{<coluna 1>}{<coluna 2>}{<linhas>}
\tabeladuas
  {Resumo de parâmetros do experimento}
  {tab:parametros}
  {Parâmetro}
  {Valor}{
    Tamanho de palavra & 16 bits \\
    Clock & 50\,MHz \\
    Nº de testes & 16 \\
    Ambiente & Simulação (Icarus/GTKWave) \\
  }

  \subsection{Tabela de Tres colunas}

% \tabelaTresColunas{<caption>}{<label>}{<col1>}{<col2>}{<col3>}{<linhas>}
\tabelatres
  {Resumo do experimento}
  {tab:resumo-exp}
  {Item}
  {Descrição}
  {Observação}{
    Clock & 50\,MHz & FPGA DE10-Lite \\
    Entradas & 4 bits (\texttt{chaves}) & Usuário \\
    Saídas & \texttt{acertou}, \texttt{errou} & LEDs \\
  }
\end{document}