\documentclass[a4paper,12pt,twoside]{article}

\usepackage[brazil]{babel}
\usepackage{poliusp} 
\usepackage{amsmath,amssymb}
\usepackage{tikz}
\usetikzlibrary{tikzmark,calc, shapes, backgrounds,arrows.meta, positioning}

% Apenas preencha os dados aqui:
\titulo{Relatório de Sistemas Digitais}
\subtitulo{Simulação do Processador LEGv8}
\Autores{
        Nome do Aluno - NUSP XXXXXXX\\
        Nome do Aluno - NUSP XXXXXXX\\
        Nome do Aluno - NUSP XXXXXXX\\
        Nome do Aluno - NUSP XXXXXXX
        }
\departamento{Departamento de Computação e Sistemas Digitais}
\data{\today}


\begin{document}
% A CAPA APARECERÁ AUTOMATICAMENTE AQUI
\tableofcontents 
\clearpage

\section{Seção com Matemática}

Texto com matemática inline: $e^{i\pi}+1=0$, $\alpha+\beta=\gamma$ e $x^2+y^2=z^2$.

\subsection{Ambiente equation}

\begin{equation}
  \int_{0}^{\infty} e^{-x^2}\,dx = \frac{\sqrt{\pi}}{2}.
\end{equation}

\subsection{Ambiente align}

\begin{align}
  (a+b)^2 &= a^2 + 2ab + b^2, \\
  (a-b)^2 &= a^2 - 2ab + b^2.
\end{align}

\section{Seção com Casos e Matrizes}

\subsection{Função definida por partes}

\begin{equation}
  f(x) =
  \begin{cases}
    x^2, & x \ge 0,\\
    -x,  & x < 0.
  \end{cases}
\end{equation}

\section{Tikz}
  \subsection{Daigrama comutativo}
    \begin{center}
    \begin{tikzpicture}[node distance=3.0cm, auto]
        % --- Definição dos Nós ---
        
        % 1. TOPO: O anel de polinômios "pai" (Abstrato)
        \node (Poly) {$\mathbb{Z}_p[x]$};
        
        % 2. DIREITA: Domínio da Frequência (Valores pontuais)
        % Aqui o \bigoplus mostra que são valores independentes
        \node (Freq) [right of=Poly, node distance=7cm] 
                     {$\displaystyle \bigoplus_{k=0}^{N-1} \mathbb{Z}_p$};
        
        % 3. BAIXO: Domínio do Tempo (Anel Negacíclico Compacto)
        % O ideal gerador é o polinômio completo
        \node (Time) [below of=Poly] {$\mathbb{Z}_p[x]/( x^N+1 )$};

        % --- As Setas ---
        
        % Seta Horizontal: Avaliação Múltipla
        % Quebrei a linha no label para não alargar demais o diagrama
        \draw[->] (Poly) -- node[align=center, above, font=\footnotesize] {\color{white}$x \mapsto(\psi^{2k+1})_{k=0}^{N-1} $} 
                            (Freq);
        
        % Seta Vertical: Projeção Canônica
        \draw[->>] (Poly) -- node[left] {\color{white}$\pi$} (Time);
        
        % Seta Diagonal: A NTT (Isomorfismo CRT)
        % Aqui mostramos que sair do anel compacto e ir para a soma direta é a NTT
        \draw[->, dashed, thick,] (Time) -- node[sloped, above] {$\mathbf{}$} 
                                           node[sloped, below] {\color{white}$\cong$} 
                                           (Freq);
    \end{tikzpicture}
    \end{center}

    \subsection{Borboleta}
    \begin{center}
    \begin{tikzpicture}[
        >=Stealth, 
        node distance=1.5cm, 
        thick,
        scale=0.75, transform shape, % Ajuste de escala para caber no slide
        op/.style={circle, draw, minimum size=0.6cm, inner sep=0pt, font=\large},
        input/.style={font=\Large},
        twiddle/.style={font=\normalsize, above, midway}
    ]

        % --- Coordenadas Verticais das Linhas ---
        \coordinate (L0) at (0, 0);
        \coordinate (L1) at (0, -2);
        \coordinate (L2) at (0, -4);
        \coordinate (L3) at (0, -6);

        % --- Entradas (Ordem Natural) ---
        \node[input] (in0) at (L0) {$x[0]$};
        \node[input] (in1) at (L1) {$x[1]$};
        \node[input] (in2) at (L2) {$x[2]$};
        \node[input] (in3) at (L3) {$x[3]$};

        % ==================================================
        % ESTÁGIO 1: Borboletas de Stride 2
        % ==================================================

        % --- Par 1 (Linhas 0 e 2) ---
        \node[op] (plus1_0) at (4, 0) {$+$};
        \node[op] (minus1_2) at (4, -4) {$-$};
        \node[op] (times1_2) at (1.5, -4) {$\times$};
        \node (psi1) at (1.5, -5) {$\psi^2$}; 

        \draw[->] (in0) -- (plus1_0);
        \draw[->] (in0) -- (minus1_2);
        \draw[->] (in2) -- (times1_2);
        \draw[->] (psi1) -- (times1_2);
        \draw[->] (times1_2) -- (minus1_2);
        \draw[->] (times1_2) -- (plus1_0);

        % --- Par 2 (Linhas 1 e 3) ---
        \node[op] (plus1_1) at (4, -2) {$+$};
        \node[op] (minus1_3) at (4, -6) {$-$};
        \node[op] (times1_3) at (1.5, -6) {$\times$};
        \node (psi2) at (1.5, -7) {$\psi^2$};

        \draw[->] (in1) -- (plus1_1);
        \draw[->] (in1) -- (minus1_3);
        \draw[->] (in3) -- (times1_3);
        \draw[->] (psi2) -- (times1_3);
        \draw[->] (times1_3) -- (minus1_3);
        \draw[->] (times1_3) -- (plus1_1);

        % ==================================================
        % ESTÁGIO 2: Borboletas de Stride 1
        % ==================================================

        % --- Topo (Resultados 0 e 1) ---
        \node[op] (plus2_0) at (8, 0) {$+$};
        \node[op] (minus2_1) at (8, -2) {$-$};
        \node[op] (times2_1) at (6, -2) {$\times$};
        \node (psi3) at (6, -3) {$\psi^1$}; 

        \draw[->] (plus1_0) -- (plus2_0);
        \draw[->] (plus1_0) -- (minus2_1);
        \draw[->] (plus1_1) -- (times2_1);
        \draw[->] (psi3) -- (times2_1);
        \draw[->] (times2_1) -- (minus2_1);
        \draw[->] (times2_1) -- (plus2_0);

        % --- Fundo (Resultados 2 e 3) ---
        \node[op] (plus2_2) at (8, -4) {$+$};
        \node[op] (minus2_3) at (8, -6) {$-$};
        \node[op] (times2_3) at (6, -6) {$\times$};
        \node (psi4) at (6, -7) {$\psi^3$};

        \draw[->] (minus1_2) -- (plus2_2);
        \draw[->] (minus1_2) -- (minus2_3);
        \draw[->] (minus1_3) -- (times2_3);
        \draw[->] (psi4) -- (times2_3);
        \draw[->] (times2_3) -- (minus2_3);
        \draw[->] (times2_3) -- (plus2_2);

        % --- Saídas (Em ordem Bit-Reversed) ---
        \node[right=0.5cm of plus2_0]  (X0) {\Large $X_0$};
        \node[right=0.5cm of minus2_1] (X2) {\Large $X_2$};
        \node[right=0.5cm of plus2_2]  (X1) {\Large $X_1$};
        \node[right=0.5cm of minus2_3] (X3) {\Large $X_3$};

        \draw[->] (plus2_0) -- (X0);
        \draw[->] (minus2_1) -- (X2);
        \draw[->] (plus2_2) -- (X1);
        \draw[->] (minus2_3) -- (X3);

    \end{tikzpicture}
    \end{center}


% Uso: \tabelaSimulacao{Legenda}{Label}{Conteúdo das Linhas}

\newpage

\subsection{Matriz e vetor}

\[
A =
\begin{pmatrix}
1 & 2 & 3\\
0 & 1 & 4\\
0 & 0 & 1
\end{pmatrix},
\qquad
b =
\begin{pmatrix}
1\\
0\\
-1
\end{pmatrix}.
\]

\section{Seção com Símbolos,Notação e Tabela}

%gera uma caixa pequena de sumario das subsection da section atual
\gerarSumarioLocal

Texto com símbolos: $\forall x \in \mathbb{R}$, $\exists y>0$ tal que $|x|<y$ e
$\sum_{k=0}^{n} k = \frac{n(n+1)}{2}$.

\[
\lim_{n\to\infty}\left(1+\frac{1}{n}\right)^n = e.
\]

\subsection{Tabela de Simulação labdig}

%exemplos de atalho para tabelas, foi pensando para uso pessoal, por isso a escolhas b
% Parâmetros: #1=Legenda, #2=Label, #3=Conteúdo das linhas
\tabelaSimulacao
{Cenário 1 -- Acerto das 16 jogadas (Simulação)}
{tab:cenario1}
{
    1  & Resetar circuito & \texttt{reset\_in = 1} & FSM no estado inicial & \\
    2  & Aguardar estabilidade & - & - & \\
    3  & Acionar iniciar & \texttt{iniciar\_in = 1} & Aguardando jogada 1 & Aguardando jogada 1 \\
    % ... adicione as outras linhas aqui ...
    19 & Acionar jogada 16 & \texttt{chaves = 4'b0100} & 
    \begin{tabular}[c]{@{}l@{}}Sinais \textbf{acertou} e \\ \textbf{pronto} ativados\end{tabular} & \\
}
\subsection{tabela de Duas Colunas}

% \tabelaDuasColunas{<caption>}{<label>}{<coluna 1>}{<coluna 2>}{<linhas>}
\tabeladuas
  {Resumo de parâmetros do experimento}
  {tab:parametros}
  {Parâmetro}
  {Valor}{
    Tamanho de palavra & 16 bits \\
    Clock & 50\,MHz \\
    Nº de testes & 16 \\
    Ambiente & Simulação (Icarus/GTKWave) \\
  }

  \subsection{Tabela de Tres colunas}

% \tabelaTresColunas{<caption>}{<label>}{<col1>}{<col2>}{<col3>}{<linhas>}
\tabelatres
  {Resumo do experimento}
  {tab:resumo-exp}
  {Item}
  {Descrição}
  {Observação}{
    Clock & 50\,MHz & FPGA DE10-Lite \\
    Entradas & 4 bits (\texttt{chaves}) & Usuário \\
    Saídas & \texttt{acertou}, \texttt{errou} & LEDs \\
  }
\end{document}