\documentclass[a4paper,12pt,twoside]{article}

\usepackage{poliusp} % Onde você colou o código acima
\usepackage{amsmath,amssymb}
% Apenas preencha os dados aqui:
\titulo{Relatório de Sistemas Digitais}
\subtitulo{Laboratório 4: Simulação do Processador LEGv8}
\Autores{Nome do Aluno - NUSP XXXXXXX}
\departamento{Departamento de Computação e Sistemas Digitais}
\data{\today}


\begin{document}
% A CAPA APARECERÁ AUTOMATICAMENTE AQUI
\tableofcontents 
\clearpage

\section{Seção com Matemática}

Texto com matemática inline: $e^{i\pi}+1=0$, $\alpha+\beta=\gamma$ e $x^2+y^2=z^2$.

\subsection{Ambiente equation}

\begin{equation}
  \int_{0}^{\infty} e^{-x^2}\,dx = \frac{\sqrt{\pi}}{2}.
\end{equation}

\subsection{Ambiente align}

\begin{align}
  (a+b)^2 &= a^2 + 2ab + b^2, \\
  (a-b)^2 &= a^2 - 2ab + b^2.
\end{align}

\section{Seção com Casos e Matrizes}

\subsection{Função definida por partes}

\begin{equation}
  f(x) =
  \begin{cases}
    x^2, & x \ge 0,\\
    -x,  & x < 0.
  \end{cases}
\end{equation}

\newpage

\subsection{Matriz e vetor}

\[
A =
\begin{pmatrix}
1 & 2 & 3\\
0 & 1 & 4\\
0 & 0 & 1
\end{pmatrix},
\qquad
b =
\begin{pmatrix}
1\\
0\\
-1
\end{pmatrix}.
\]

\section{Seção com Símbolos e Notação}

Texto com símbolos: $\forall x \in \mathbb{R}$, $\exists y>0$ tal que $|x|<y$ e
$\sum_{k=0}^{n} k = \frac{n(n+1)}{2}$.

\[
\lim_{n\to\infty}\left(1+\frac{1}{n}\right)^n = e.
\]

\end{document}